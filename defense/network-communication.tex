\begin{frame}
  \frametitle{\emoji{puzzle-piece} \textit{CNI plugins}}
  \vspace{.5em}
  \begin{block}{\emoji{memo} \emph{Def.}}
    \begin{quote}{Red Hat sysadmin's guide}
      \vspace{.5em}
      \small
      A CNI plugin is responsible for \alert{\textbf{inserting a network interface}} into the
      \alert{\textbf{container network namespace}} (e.g., one end of a virtual ethernet (veth)
      pair) and making any \alert{\textbf{necessary changes}} on the host (e.g., attaching the
      other end of the veth into a bridge).
    \end{quote}
  \end{block}
  \pause
  \begin{figure}
    \centering
    \includegraphics[height=.5\textheight]{defense/img/CNI-generic}
  \end{figure}
\end{frame}

\begin{frame}
  \frametitle{\emoji{exploding-head}  \textit{``And making any necessary changes
      on the host $\dots$''}}
  \pause
  \begin{columns}
    \column{.4\textwidth}
    \only<2>{\emoji{backhand-index-pointing-right} \alert{\textbf{\textit{Calico}}}}
    \only<2>{
      \begin{itemize}
        \itemsep0em
        \item Based on \alert{\textbf{BGP} routing} protocol
        \item \textit{Felix}: agent which configures the network
        \item \textit{BIRD}: BGP client for routing updates
        \item \textit{confd}: ensure the configuration is up-to-date
      \end{itemize}
    }

    \only<2>{\qquad\texttt{Flannel}}

    \only<2>{\qquad\texttt{Cilium}}

    \only<3->{\qquad\texttt{Calico}}

    \only<3>{\emoji{backhand-index-pointing-right} \alert{\textbf{\textit{Flannel}}}}
    \only<3>{
      \begin{itemize}
        \itemsep0em
         \item Based on \alert{\textbf{VXLAN} encapsulation}
         \item Runs a \alert{flanneld} binary agent responsible for
           managing the network configuration, IP address management and subnets
      \end{itemize}
    }

    \only<3>{\qquad\texttt{Cilium}}

    \only<4->{\qquad\texttt{Flannel}}

    \only<4->{\emoji{backhand-index-pointing-right} \alert{\textbf{\textit{Cilium}}}}
    \only<4>{
      \begin{itemize}
        \itemsep0em
        \item Based on \alert{\textbf{eBPF}} (extended Berkeley Packet Filter)
        \item Runs a agent which configures the network
        \item similar to a JIT compilation in a sand-boxed environment inside
          the \alert{kernel space}
      \end{itemize}
    }
    \column{.66\textwidth}
    \begin{figure}
      \centering
      \only<2>{\includegraphics[width=\textwidth]{defense/img/calico-cni}}
      \ \vfill
      \only<2>{\includegraphics[width=.25\textwidth]{img/cnis/calico-logo}}
      \pause
      \only<3>{\includegraphics[width=\textwidth]{defense/img/flannel-cni}}
      \ \vfill
      \only<3>{\includegraphics[width=.25\textwidth]{img/cnis/flannel-logo}}
      \pause
      \only<4>{\includegraphics[width=\textwidth]{defense/img/cilium-cni}}
      \ \vfill
      \only<4>{\includegraphics[width=.25\textwidth]{img/cnis/cilium-logo}}
    \end{figure}
  \end{columns}
\end{frame}
